%test.tex
\documentclass{article}
\usepackage{CJKutf8}
\usepackage{amsmath}
\begin{CJK}{UTF8}{gbsn}
\title{拉格朗日对偶与KKT条件}
\author{布衣之莠}
\begin{document}
\maketitle
\section{问题提出:}
\subsection{问题1:}
有下面最优化问题:
\begin{equation}
\min f(x) \tag{1.1.1}
\end{equation}
\begin{equation}
s.t. \quad h_j(x)=0\tag{1.1.2}
\end{equation} 
\begin{equation}
g_k(x) \le 0  \tag{1.1.3}
\end{equation}
\\
\\
\subsection{问题2:}
在问题1的情况下,定义
\begin{equation}
L(x,\mu,\lambda)=f(x)+\sum_j\mu_jh_j(x)+\sum_k\lambda_kg_k(x) \tag{1.2.1}
\end{equation}
其中$\mu_j$任意取值,$\lambda_k \ge 0$,那么,把$L$看做$x,\mu,\lambda$的三元函数,将有
\begin{equation}
f(x)=\max_{\mu,\lambda;\lambda_k \ge 0}L(x,\mu,\lambda) \tag{1.2.2}
\end{equation}
所以问题1的解决可由下式给出:
\begin{equation}
\min_x f(x)= \min_x \max_{\mu,\lambda;\lambda_k\ge 0}L(x,\mu,\lambda) \tag{1.2.3}
\end{equation}
\\
\\
\subsection{问题3:}
定义函数$J(x,\mu,\lambda)$,那么有下面弱对偶性质(weak duality):
\begin{equation}
\max_{\mu,\lambda}\min_xJ\le \min_x\max_{\mu,\lambda}J\ \tag{1.3.1}
\end{equation}
\section{问题讨论:}
\subsection{先讨论问题1和问题2之间的关系:}
问题2中只有1个约束就是$\lambda_k \ge 0$。那么怎么得出(1.2.2)呢?\\
考虑
\begin{equation}
\max_{\mu,\lambda;\lambda_k\ge 0}L(x,\mu,\lambda)=\max_{\mu,\lambda;\lambda_k\ge 0}[f(x)+\sum_j\mu_jh_j(x)+\sum_k\lambda_kg_k(x)] \tag{2.1.1}
\end{equation}

如果最大值存在,那么公式(2.1.1)中$h_j(x)=0$,否则由于$\mu_j$的任意性,总能使$L(x,\mu_j,\lambda_k)\rightarrow \infty$(只要取$\mu_j$和$h_j(x)$同号且足够大)。而且有$g_k(x)\le 0$,否则若$g_k(x)>0$,在符合约束条件下取$\lambda_k$无限大,总能使$L(x,\mu_j,\lambda_k)\rightarrow \infty$,所以公式2.1.1的第三项小于等于0,考虑到求$L(x,\mu,\lambda)$最大值,第三项只有取零。这样一来,可以看到如果公式(2.1.1)存在最大值,那么第二项和第三项都是0,即可得到
\begin{equation}
f(x)=\max_{\mu,\lambda;\lambda_k \ge 0}L(x,\mu,\lambda) \tag{2.2.2}
\end{equation}
这样问题1就转化到问题2,可以得到公式(1.2.3)
\\
\subsection{单独讨论问题3:}
令
\begin{equation}
D_1(\mu,\lambda)=\min_xJ(x,\mu,\lambda) \tag{2.2.1}
\end{equation}
\begin{equation}
D_2(x)=\max_{\mu,\lambda}J(x,\mu,\lambda) \tag{2.2.2}
\end{equation}
如果上面两式定义的最值都存在,根据上面两式定义:
\begin{equation}
D_1(\mu,\lambda) \le J(x,\mu,\lambda) \le D_2(x) \tag{2.2.3}
\end{equation}
公式(2.2.3)表面对于任意的$\mu,\lambda$,任意的$x$都有$D_1(\mu,\lambda) \le L(x,\mu,\lambda) \le D_2(x)$,所以有下面:
\begin{equation}
\max_{\mu,\lambda}D_1(\mu,\lambda) \le \min_{x}D_2(x) \tag{2.2.4}
\end{equation}
即:
\begin{equation}
\max_{\mu,\lambda}\min_xJ\le \min_x\max_{\mu,\lambda}J\ \tag{2.2.5}
\end{equation}
这样弱对偶性质就解释了。为了说明问题3和问题1,2无关,我们用$J$代替$L$,下面我们替代回去,即:
\begin{equation}
\max_{\mu,\lambda}\min_xL\le \min_x\max_{\mu,\lambda}L\ \tag{2.2.5}
\end{equation}\\
\subsection{问题3的结论应用于解决问题2,间接也就解决问题1:}
求解原问题公式(1.2.3)的对偶形式为:
\begin{equation}
\max_{\mu,\lambda;\lambda_k \ge 0}D_1(\mu,\lambda)=\max_{\mu,\lambda;\lambda_k \ge 0}\min_{x}L(x,\mu,\lambda) \tag{2.3.1}
\end{equation}
我们讨论问题2时只是给出了原问题(1.2.3)的下界是公式(2.3.1)。
引入强对偶性,强对偶性是指公式(2.2.5)取等号,这样求解原问题就完全转化为求解对偶问题了。\\
\textbf{slater条件:}(1)原始问题为凸优化问题(2)$\exists x \quad  s.t. \quad g_k(x)<0$
\\对于问题1,如果slatter条件成立,那么满足强对偶性,这是强对偶的充分条件。\\
对于问题1,假设$x^*$是原问题(不一定是凸)最优解,$\mu^*,\lambda^*$是对偶问题的最优解
那么,如果强对偶性成立,有下式
\begin{equation}
\begin{split}
f(x^*)=D_2(x^*)=D_1(\mu^*,\lambda^*)\qquad\\
	 =\min_{x}f(x)+\sum_j\mu_j^*h_j(x)+\sum_k\lambda_k^*g_k(x)\\
	 \le f(x^*)+\sum_j\mu_j^*h_j(x^*)+\sum_k\lambda_k^*g_k(x^*)\le f(x^*)
\end{split}
\end{equation}
分析上式,第一个等号成立来源于问题(2)的结论,第二个等号成立源于满足强对偶性,第三个等号源于$\mu^*,\lambda^*$是$D_1(\mu,\lambda)$的解,第一个$\le$是源于$D_1(\mu,\lambda)$的定义,第二个$\le$是因为问题1的约束条件$h_j(x)=0,g_j(x) \le 0$以及讨论问题2给出的$\lambda_k \ge 0$。所以公式(1)不等号可以全换成等号,因为$x^*$是$D_2(x)$的极值点,所以有
\begin{equation}
\nabla_{x^*}L(x,\mu^*,\lambda^*)=0  \tag{2.3.2}
\end{equation}
根据公式(1)第二个不等号以及$h_j(x^*)=0$,有:
\begin{equation}
\lambda_k^*g_k(x^*)=0 \tag{2.3.3}
\end{equation}
和问题1给出的约束条件联合起来,得到kkt条件:
\begin{equation}
\nabla_{x}L(x,\mu,\lambda)=0  \tag{2.3.4}
\end{equation}
\begin{equation}
\lambda_kg_k(x)=0 \tag{2.3.5}
\end{equation}
\begin{equation}
h_j(x)=0 \tag{2.3.6}
\end{equation}
\begin{equation}
g_k(x) \le 0 \tag{2.3.7}
\end{equation}
\begin{equation}
\lambda_k \ge 0 \tag{2.3.8}
\end{equation}
这样我们在满足强对偶性时推出kkt条件,也就是说kkt条件是强对偶性的必要条件。\\
当原始问题是凸优化问题时,设$\mu^*,\lambda^*$是$D_1$的解,$x^*$是$D_2$的解,结合kkt条件可以推出强对偶性。如下:
\begin{equation}
D_1(\mu^*,\lambda^*)=\min_x(x,\mu,\lambda)=L(x^*,\mu^*,\lambda^*)
=f(x^*)+\sum_j\mu_j^*h_j(x^*)+\sum_k\lambda_kg_k^*(x^*)=f(x^*) \tag{2.3.9}
\end{equation}
分析上式,第一个等号是因为$\mu^*,\lambda^*$是$D_1$的解,第二个等式是因为$x^*$是$D_2$的解,第三个等式是定义,第四个是带入kkt条件。\\
对于问题1,以上说明了kkt条件是强对偶性的充要条件。
\\\\
\section{参考资料:}
https://www.cnblogs.com/ooon/p/5723725.html
\end{CJK}
\end{document}