
%test.tex
\documentclass{article}
\usepackage{CJKutf8}
\usepackage{amsmath}
\begin{CJK}{UTF8}{gbsn}
\title{EM算法与GMM模型}
\author{布衣之莠}
\begin{document}
\maketitle
\section{GMM模型:}
\subsection{问题引入:}
已知N个人身高$\{y_1,y_2,...,y_N\}$,要求解人群中身高的分布,人群中有男有女,男女符合不同分布,但是事先我们并不知道某个人服从哪种分布。我们用混合高斯模型,假设男女的分布各符合一个高斯分布,用极大似然求解这两个分布的参数。
\subsection{问题建模:}
建立高斯混合模型,即符合下式的概率分布模型:
\begin{equation}
P(y|\theta)=\sum_k\alpha_k\phi(y|\theta_k)  \tag{1.2.1}
\end{equation}
其中,$\alpha_k$是系数,$\alpha_k \ge 0,\sum_k\alpha_k=1$,$\phi(y|\theta_k)$是高斯分布密度,成为第k个分模型,参数是$\theta_k=(\mu_k,\sigma_k^2)$
\begin{equation}
\phi(y|\theta_k)=\frac{1}{\sqrt{2\pi}\sigma_k}exp(-\frac{(y-\mu_k)^2}{2\sigma_k^2}) \tag{1.2.2}
\end{equation}
对于我们上面引入的问题,显然k=1,2,似然函数是:
\begin{equation}
L(\theta)=\sum_j \log p(y_j|\theta)=\sum_j \log \sum_k \alpha_k p(y_j|\theta_k) \tag{1.2.3}
\end{equation}
以上似然函数是对和求对数,很难优化,所以我们采用EM算法去求解。
\section{EM算法:}
考虑下面似然函数的优化问题:
\begin{equation}
L(\theta)=\sum_j \log p(y_j|\theta)=\sum_j \log \sum_i p(y_j,z_i|\theta) 
\tag{2.1.1}
\end{equation}
先介绍Jensan不等式。\\
对于凸函数$f(x)$,满足:
\begin{equation}
E(f(x)) \ge f(E(x)) \tag{2.1.2}
\end{equation}
对于凹函数$f(x)$,满足:
\begin{equation}
E(f(x)) \le f(E(x)) \tag{2.1.3}
\end{equation}
取等号是$f(x)=C$,即是常数,Jensen不等式意义在于描述了函数值的期望与期望的函数值关系,如果我们取$f(x)$是$\log$,
我们看到上面式子(2.1.3)右边是对和求对数,而左边是先内部求对数外部再求和,利用这一点我们可以解决在问题1中遇到的优化难点:对和求对数。
我们在(2.1.1)中引入$Q_j(z_i)$,所以:
\begin{equation}
L(\theta)=\sum_j \log \sum_i Q_j(z_i) \frac{p(y_j,z_i|\theta)}{Q_j(z_i)} \ge 
\sum_j \sum_i Q_j(z_i) \log \frac{p(y_j,z_i|\theta)}{Q_j(z_i)} \tag{2.1.4}
\end{equation}
上面引入的$Q(z)$是关于$z$的某种分布,满足:
\begin{equation}
\sum_iQ_j(z_i)=1 \tag{2.1.5}
\end{equation}
(2.1.4)中不等号利用了Jensen不等式。如果可以取等号,原来对和求对数的难题就解决了,可以转化为先求对数在求和。等号成立,由Jensen不等式,
\begin{equation}
\frac{p(y_j,z_i|\theta}{Q_j(z_i)}=C \tag{2.1.6}
\end{equation}
由(2.1.5),(2.1.6)可以得出:
\begin{equation}
\sum_ip(y_j,z_i|\theta)=C \tag{2.1.7}
\end{equation}
所以,
\begin{equation}
Q_j(z_i)=\frac{p(y_j,z_i|\theta)}{C}=\frac{p(y_j,z_i|\theta)}{\sum_ip(y_j,z_i| \theta)} \tag{2.1.8}
\end{equation}
(2.1.8)给出了我们引入的$Q_j(z_i)$的表达式。\\
EM算法:\\
选取初始值$\theta^0$初始化$\theta$,t=0 \\
Repeat \{\\
E step:\\
\begin{equation}
Q_j^t(z_i)=\frac{p(y_j,z_i|\theta^t)}{\sum_ip(y_j,z_i|\theta^t)}=\frac{p(y_j,z_i|\theta^t)}{p(y_j|\theta^t)}=p(z_i|y_j,\theta^t) \tag{2.1.9}
\end{equation}
M step:
\begin{equation}
\theta^{t+1} = arg \max_{\theta} \sum_j \sum_i Q_j^t(z_i) \log \frac{p(y_j,z_i|\theta)}{Q_j^t(z_i)}  \tag{2.1.10}
\end{equation}
\begin{equation}
t = t+1 \tag{2.1.11}
\end{equation}
\}直到收敛!

\section{EM算法求解GMM模型:}
把上面的EM算法应用于求解我们第一步提出的问题。\\
\begin{equation}
L(\theta)=\sum_j \log p(y_j| \theta)=\sum_j \log \sum_k \alpha_k p(y_j|\theta_k)=\sum_j \log \sum_k \gamma_{jk} \frac{\alpha_k p(y_j|\theta_k)}{\gamma_{jk}} \tag{3.1.1}
\end{equation}
由Jensen不等式:
\begin{equation}
L(\theta) \ge \sum_j \sum_k \gamma_{jk} \log \frac{\alpha_k p(y_j|\theta_k)}{\gamma_{jk}} \tag{3.1.2}
\end{equation}
利用EM算法可以知道当
\begin{equation}
\gamma_{jk}=\frac{p(y_j,z=k|\theta)}{\sum_{k}p(y_j,z=k|\theta)}=\frac{\alpha_k \phi(y_j|\theta_k)}{\sum_k \alpha_k \theta(y_j|\theta_k)}  \tag{3.1.3}
\end{equation}
(3.1.2)中等号成立。
我们将高斯分布密度函数带入公式(3.1.2),有
\begin{equation}
L(\theta)=\sum_j \sum_k \gamma_{jk} \log \frac{p(y_j|\theta_k)}{\gamma_{jk}}=\sum_j \sum_k \gamma_{jk} \log \frac{\frac{1}{\sqrt{2\pi}\sigma_k}exp(-\frac{(y-\mu_k)^2}{2\sigma_k^2})}{\gamma_{jk}}  \tag {3.1.4}
\end{equation}
化简:
\begin{equation}
L(\theta)=\sum_j \sum_k \gamma_{jk}[\log \alpha_k+\log(\frac{1}{\sqrt{2\pi}})-\log \sigma_k-\frac{(y-\mu_k)^2}{2\sigma_k^2}-\log \gamma_{jk}] \tag {3.1.5}
\end{equation}
在约束$\sigma_k \alpha_k=1$下对(3.1.5)进行最小化,可以利用lagrange乘子法求解此约束优化问题。
令:
\begin{equation}
H(\theta)=L(\theta)+\lambda(\sum_k\alpha_k-1)   \tag{3.1.6}
\end{equation}
\begin{equation}
\frac{\partial{H(\theta)}}{\partial{\alpha_k}}=0 \tag{3.1.7}
\end{equation}
\begin{equation}
\frac{\partial{H(\theta)}}{\partial{\lambda}}=0 \tag{3.1.8}
\end{equation}
\begin{equation}
\frac{\partial{H(\theta)}}{\partial{\mu_k}}=0 \tag{3.1.9}
\end{equation}
\begin{equation}
\frac{\partial{H(\theta)}}{\partial{\sigma_k}}=0 \tag{3.1.10}
\end{equation}
可以推出公式:
\begin{equation}
\alpha_k = \frac{\sum_j \gamma_{jk}}{N} \tag{3.1.11}
\end{equation}
\begin{equation}
\mu_k = \frac{\sum_j \gamma_{jk}y_j}{\sum_j \gamma_{jk}} \tag{3.1.12}
\end{equation}
\begin{equation}
\sigma_k^2 = \frac{\sum_j \gamma_{jk}(y_j-\mu_k)^2}{\sum_j \gamma_{jk}} \tag{3.1.13}
\end{equation}
所以训练GMM模型的算法步骤如下:
选取初始化值初始化$\theta$ \\
Repeat \{\\
(1) 
\begin{equation}
\gamma_jk = \frac{\alpha_k \phi(y_j|\theta_k)}{\sum_k \alpha_k \theta(y_j|\theta_k)}
\end{equation}
(2)
根据$\gamma_{jk}$和式(3.1.11)(3.1.12)(3.1.13)估计每个分模型的参数。\\
\}\\
\section{参考资料:}
https://www.cnblogs.com/mindpuzzle/archive/2013/04/05/2998746.html \\
https://www.cnblogs.com/mindpuzzle/archive/2013/04/24/3036447.html \\
https://www.cnblogs.com/txg198955/p/4097543.html
\end{CJK}
\end{document}